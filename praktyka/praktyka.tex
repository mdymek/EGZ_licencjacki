%! Suppress = TooLargeSection
%! Suppress = SentenceEndWithCapital
%! Suppress = LineBreak
%! Suppress = MissingLabel
%! Suppress = Unicode
\documentclass[12pt]{article}
\usepackage{amsmath}
\usepackage{amsfonts}
\usepackage[colorinlistoftodos]{todonotes}
\usepackage[utf8]{inputenc}
\usepackage{lmodern}
\usepackage[MeX]{polski}
\usepackage{rotating}
\usepackage{geometry}
\usepackage{multirow}
\usepackage{xcolor,colortbl}
\usepackage{xparse}
\usepackage[most]{tcolorbox}
\usepackage{fancyvrb,newverbs,xcolor}
\usepackage{amssymb}
\usepackage{booktabs}
\usepackage{float}
\usepackage{graphicx}
\graphicspath{{../graphics/praktyka/}}
\usepackage{minted}
\usepackage{tikz}
\usetikzlibrary{automata,arrows,positioning,calc}
\usepackage{forest}
\usepackage{hyperref}
\usepackage{enumitem}
\usepackage{diagbox}
\usepackage{subfiles}

\hypersetup{
colorlinks=true,
linkcolor=black,
filecolor=magenta,
urlcolor=cyan,
}

\urlstyle{same}

\NewDocumentCommand{\newframedtheorem}{O{}momo}{%
\IfNoValueTF{#3}
{%
\IfNoValueTF{#5}
{\newtheorem{#2}{#4}}
{\newtheorem{#2}{#4}[#5]}%
}
{\newtheorem{#2}[#3]{#4}}
\tcolorboxenvironment{#2}{#1}%
}

\newframedtheorem{theorem}{Twierdzenie}[section]
\newframedtheorem{definition}[theorem]{Definicja}
\newframedtheorem{exercise}[theorem]{Zadanie}


\renewcommand{\figurename}{Załącznik}

\begin{document}

    \begin{titlepage}

        \newcommand{\HRule}{\rule{\linewidth}{0.5mm}}

        \center

        \textsc{\LARGE Uniwerystet Jagielloński}\\[1.5cm]

        \HRule \\[0.4cm]
        %        { \huge \bfseries ETL}\\[0.4cm]
        \textsc{\Large Pytania do egzaminu licencjackiego}\\[0.4cm]
        \textsc{\large na kierunku Informatyka}\\[0.4cm]
        \textsc{\Large Praktyka}\\[0.4cm]
        \HRule \\[1.0cm]

        \Large %\emph{Autorzy:}\\
        Karol \textsc{Chrząstek}\\
        Małgorzata \textsc{Dymek}\\
        Michał \textsc{Piotrowski}\\
        Dominik \textsc{Wołek}\\
        Dagmara \textsc{Zdybał}\\[1.5cm]

        \includegraphics[scale=0.27]{uj.jpg}\\[2cm]

        {\large Rok akademicki 2019/2020}\\

        \vfill
    \end{titlepage}

    \tableofcontents

    \newpage

    \begin{center}{\LARGE Matematyczne podstawy informatyki}\end{center}

    \subfile{mpi}

    \begin{center} {\LARGE Teoretyczne podstawy informatyki} \end{center}

    \subfile{tpi}

    \begin{center}{\LARGE Wytwarzanie oprogramowania}\end{center}

    \subfile{wo}

    \begin{center}{\LARGE Inżynieria systemów}\end{center}

    \subfile{is}

\end{document}
