%! Suppress = MissingImport
%! Suppress = TooLargeSection
%! Suppress = SentenceEndWithCapital
%! Suppress = LineBreak
%! Suppress = MissingLabel
%! Suppress = Unicode

\documentclass[main.tex]{subfiles}

\begin{document}

    \section{Reprezentacja liczb całkowitych; arytmetyka.}
    \section{Reprezentacja liczb rzeczywistych; arytmetyka zmiennopozycyjna.}
    \section{Różnice w wywołaniu funkcji statycznych, niestatycznych i wirtualnych w C++.}
    \section{Sposoby przekazywania parametrów do funkcji (przez wartość, przez referencję). Zalety i wady.}
    \section{Wskaźniki, arytmetyka wskaźników, różnica między wskaźnikiem a referencją w C++.}
    \section{Podstawowe założenia paradygmatu obiektowego: dziedziczenie, abstrakcja, enkapsulacja, polimorfizm.}
    \section{Funkcje zaprzyjaźnione i ich związek z przeładowaniem operatorów w C++.}
    \section{Programowanie generyczne na podstawie szablonów w języku C++.}
    \section{Podstawowe kontenery w STL z szerszym omówieniem jednego z nich.}
    \section{Obsługa sytuacji wyjątkowych w C++.}
    \section{Obsługa plików w języku C.}
    \section{Model wodospadu a model spiralny wytwarzania oprogramowania.}
    \section{Diagram sekwencji i diagram przypadków użycia w języku UML.}
    \section{Klasyfikacja testów.}
    \section{Model Scrum: struktura zespołu, proces wytwarzania oprogramowania, korzyści modelu.}
    \section{Wymagania w projekcie informatycznym: klasyfikacja, źródła, specyfikacja, analiza.}
    \section{Analiza obiektowa: modele obiektowe i dynamiczne, obiekty encjowe, brzegowe i sterujące.}
    \section{Wzorce architektury systemów.}

\end{document}