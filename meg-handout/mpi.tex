%! Suppress = MissingImport
%! Suppress = TooLargeSection
%! Suppress = SentenceEndWithCapital
%! Suppress = LineBreak
%! Suppress = MissingLabel
%! Suppress = Unicode

\documentclass[main.tex]{subfiles}

\begin{document}

    \section{Zasada indukcji matematycznej.}

    \begin{itemize}
        \item Dla $T(n)$ będącego funkcją/formą zdaniową jeżeli (1) zachodzi $T(0)$, (2) $\forall n \in \mathbb{N} ~~ T(n) \Rightarrow T(n+1)$
        to $T(n)$ jest prawdziwa dla każdego $n \in \mathbb{N}$.

        \item W dowodzie udowadniamy że $M = \{ n \in \mathbb{N}: ~ T(n) ~ zachodzi \}$ jest równe $\mathbb{N}$ przy pomocy
        \textbf{Aksjomatu 5 liczb naturalnych (Peano)} -- zbiór liczb naturalnych taki, że (1) $1$ jest elementem
        tego zbioru, (2) dla każdej liczby ze zbioru należy do niego również jej następnik, zawiera wszystkie liczby naturalne.
    \end{itemize}


    \section{Porządki częściowe i liniowe. Elementy największe, najmniejsze, maksymalne i minimalne.}
    \begin{itemize}
        \item Relacja \textbf{częściowego porządku} ($\leqslant, \preceq, \prec$) to relacja \textbf{zwrotna} ($\forall x \in X ~ xRx$),
        \textbf{przechodnia} ($\forall x,y,z \in X ~ xRy \wedge yRz \Rightarrow xRz$) i \textbf{antysymetryczna}
        ($\forall x,y \in X ~ xRy \wedge yRx \Rightarrow x = y$.

        \begin{itemize}
            \item \textbf{Elementy porównywalne} to takie, które są w relacji ze sobą.
            \item \textbf{Element maksymalny}, jeśli $~~ \forall a \in A ~~ m \preceq a ~ \Rightarrow  ~ a = m$.
            \item \textbf{Element minimalny}, jeśli $~~ \forall a \in A ~~ a \preceq m ~ \Rightarrow  ~ a = m$.
            \item \textbf{Element największy}, jeśli $~~ \forall a \in A ~~ a \preceq m$.
            \item \textbf{Element najmniejszy}, jeśli $~~ \forall a \in A ~~ m \preceq a$.
        \end{itemize}

        \item \textbf{Diagram Hassego} - graf skierowany przedstawiający częściowy porządek w zbiorze, w odpowiedni sposób
        przedstawiony graficznie.

        \item \textbf{Relacja liniowego porządku} to relacja \textbf{zwrotna}, \textbf{przechodnia}, \textbf{antysymetryczna}
        i \textbf{spójna} ($\forall x,y \in X ~ xRy \vee yRx \vee x = y$).
    \end{itemize}


    \section{Relacja równoważności i zbiór ilorazowy.}

    \begin{itemize}
        \item \textbf{Relacja równoważości} jest \textbf{zwrotna}, \textbf{symetryczna} i \textbf{przechodnia}.

        \item \textbf{Klasa abstrakcji} elementu x (wzgledem relacji R) to: $[x]_{R} = \{y \in X: xRy\}$. Element x to
        reprezentant klasy abstrakcji $[x]_{R}$.

        \item \textbf{Zbiór ilorazowy} zbioru X przez relację R to zbiór wszystkich klas abstrakcji.
    \end{itemize}


    \section{Metody dowodzenia twierdzeń: wprost, nie wprost, przez kontrapozycję.}

    Dla prawdziwości zdania:
    \begin{align*}
        p \Rightarrow q
    \end{align*}

    \textbf{Dowód wprost}: $ p \Rightarrow q$.

    \textbf{Dowód nie wprost}: $(p \Rightarrow q) \Leftrightarrow (\neg q \Rightarrow \neg p)$ (prawo \textbf{kontrapozycji}).

    \textbf{Dowód przez zaprzeczenie}: $(p \Rightarrow q) ~ \Leftrightarrow ~ (\neg p \vee q) ~ \Leftrightarrow ~ \neg(p \wedge \neg q)$


    \section{Metody numeryczne rozwiązywania równań nieliniowych: bisekcji, siecznych, Newtona.}

    \subsection{Metoda połowienia (bisekcji)}

    \textbf{Założenia:} $f$ jest funkcją ciągłą w przedziale $[a,b]$, $f(a)f(b) < 0$.

    \noindent Z \textbf{własności Darboux} funkcji ciągłych, funkcja f ma miejsce zerowe w przedziale $[a,b]$.

    \begin{definition}
        \textbf{Algorytm bisekcji} polega na obliczeniu $f(c_k)$, gdzie $c_k = \frac{a_k + b_k}{2}$ i zastąpieniu przez
        $c_k$ tej z liczb $a_k, b_k$ dla której funkcja $f$ ma taki sam znak.
        \begin{center}
            $(a_{k+1}, b_{k+1}) = (c_k, b_k)$ jeżeli $f(a_k)f(c_k) > 0$\\
            $(a_{k+1}, b_{k+1}) = (a_k, c_k)$ jeżeli $f(b_k)f(c_k) > 0$
        \end{center}
        Jeżeli $f(c_k) = 0$ to kończymy obliczenia.
    \end{definition}
    \begin{itemize}
        \item Metoda bisekcji jest \textbf{niezawodna}, ale \textbf{wolno zbieżna}.
        \item Kryteria zakończednia obliczeń: osiągnięta \textbf{dokładność}, \textbf{bliskość} wartości funkcji do 0,
        ilość \textbf{iteracji}.
        \item \textbf{dokładność} przybliżenia w k-tej iteracji: $|x_k - \alpha| \leq \frac{|b - a|}{2^k}$.
    \end{itemize}


    \subsection{Metoda siecznych}

    \textbf{Założenia:} $f$ jest funkcją ciągłą w przedziale $[a,b]$, $f(a)f(b) < 0$.

    \begin{align*}
        x_{k+1} = x_k - \frac{f(x_k)(x_k - x_{k-1})}{f(x_k) - f(x_{k-1})}
    \end{align*}
    gdzie $x_0 = a$, $x_1 = b$.

    \begin{itemize}
        \item W metodzie siecznych rezygnujemy z założenia, że funkcja na końcach przedziału ma różne znaki.
        \item \textbf{Może się zdarzyć, że metoda wyprodukuje ciąg rozbieżny.}
        \item \textbf{Szybsza zbieżność} ciągu iteracji, jeśli zaczynamy od dobrego przybliżenia.
    \end{itemize}

    \subsection{Metoda Newtona (stycznych)}

    \begin{align*}
        x_{n+1} = x_n - \frac{f(x_n)}{f'(x_n)}
    \end{align*}

    \textbf{Zmodyfikowana metoda Newtona:}
    \begin{align*}
        x_{n+1} = x_n - \frac{f'(x_n) \pm \sqrt{(f'(x_n))^2 - 2f(x_n)f''(x_n)}}{f''(x_n)}
    \end{align*}


    \begin{itemize}
        \item Metoda Newtona w wielu sytuacjach daje \textbf{bardzo szybką zbieżność}.
        \item Podstawową wadą tej metody jest konieczność obliczenia pochodnej funkcji.
        \item Zmodyfikowana metoda Newtona daje bardzo szybką zbieżność, jeśli $x_n$ jest już dobrym przybliżeniem
        pierwiastka.
        \item Koszt wyznaczenia kolejnego przybliżenia w zmodyfikowanej metodzie może przerastać korzyści płynące
        z szybszej zbieżności.
    \end{itemize}

    \textbf{Wielowymiarowa metoda Newtona}.\\

    Jeśli $f = (f_1, f_2, \dots, f_n): \mathbb{R}^n \rightarrow \mathbb{R}^n$ jest różniczkowalną funkcją wielu zmiennych
    możemy przybliżyć ją lokalnie:

    \begin{align*}
        f(x) \approx f(x_0) + Df(x_0)(x - x_0)
    \end{align*}

    gdzie

    \begin{align*}
        Df(x_0) =
        \begin{bmatrix}
            \frac{\delta f_1}{\delta x_1}(x_0) & \frac{\delta f_1}{\delta x_2}(x_0) & \dots & \frac{\delta f_1}{\delta x_n}(x_0)\\
            \frac{\delta f_2}{\delta x_1}(x_0) & \frac{\delta f_2}{\delta x_2}(x_0) & \dots & \frac{\delta f_2}{\delta x_n}(x_0)\\
            \dots & \dots & \dots & \dots\\
            \frac{\delta f_n}{\delta x_1}(x_0) & \frac{\delta f_n}{\delta x2}(x_0) & \dots & \frac{\delta f_n}{\delta x_n}(x_0)
        \end{bmatrix}
    \end{align*}

    Rozwiązując równanie $f(x_0) + Df(x_0)(x - x_0) = 0$ otrzymujemy:
    \begin{align*}
        x^{(i+1)} = x^{(i)} - [Df(x^{(i)})]^{-1} f(x^{(i)})
    \end{align*}
    \begin{align*}
    [Df(x^{(i)})]
        ( x^{(i+1)} - x^{(i)} ) = -f(x^{(i)})
    \end{align*}


    \section{Rozwiązywanie układów równań liniowych: metoda eliminacji Gaussa, metody iteracyjne Jacobiego i Gaussa-Seidla.}

    \subsection{Metoda eliminacji Gaussa}

    Obliczając rząd macierzy metodą Gaussa należy za pomocą operacji elementarnych na wierszach sprowadzić macierz do
    macierzy schodkowej. Wtedy wszystkie niezerowe wiersze są liniowo niezależne.

    \hfill \\
    \begin{center}{\large Metody iteracyjne}\end{center}
    Ogólna postać metody iteracyjnej:
    \begin{align*}
        Ax = b
    \end{align*}
    \begin{align*}
        Qx^{n+1} = (Q - A)x^n + b = \tilde{b}
    \end{align*}

    \begin{align*}
        \begin{bmatrix}
            5 & -2 & 3\\
            2 & 4 & 2\\
            2 & -1 & -4\\
        \end{bmatrix}
        \begin{bmatrix}
            x_1\\
            x_2\\
            x_3\\
        \end{bmatrix}
        =
        \begin{bmatrix}
            10\\
            0\\
            0\\
        \end{bmatrix}
    \end{align*}

    \subsection{Metoda iteracyjna Jacobiego}

    \subsubsection{Algebraicznie}
    \begin{align*}
        \left\{\begin{matrix}
                   x^{N+1}_1 = \frac{1}{5}(10 + 2x^N_2 - 3x^N_3)\\
                   x^{N+1}_2 = \frac{1}{4}(-2x^N_1 - 2x^N_3)\\
                   x^{N+1}_3 = -\frac{1}{4}(x^N_2 - 2x^N_1)
        \end{matrix}\right.
    \end{align*}

    \subsubsection{Macierzowo}
    \begin{align*}
        Q = D ~~ \text{(diagonalna)}
    \end{align*}

    \subsection{Metoda iteracyjna Gaussa-Seidla}
    \subsubsection{Algebraicznie}
    \begin{align*}
        \left\{\begin{matrix}
                   x^{N+1}_1 = \frac{1}{5}(10 + 2x^N_2 - 3x^N_3)\\
                   x^{N+1}_2 = \frac{1}{4}(-2x^N_1 - 2x^{\mathbf{N+1}}_3)\\
                   x^{N+1}_3 = -\frac{1}{4}(x^{N+1}_2 - 2x^{\mathbf{N+1}}_1)
        \end{matrix}\right.
    \end{align*}

    \subsubsection{Macierzowo}
    \begin{align*}
        Q = L + D ~~ \text{(diagonalna i dolnotrójkątna)}
    \end{align*}

    \newpage


    \section{Wartości i wektory własne macierzy: numeryczne algorytmy ich wyznaczania.}
    \begin{definition}
        Jeżeli $\mathbf{Ax = \lambda x}$ dla $x \neq 0$ to $\lambda$ jest \textbf{wartością własną} A, a x -
        \textbf{wektorem własnym} A odpowiadającym $\lambda$.
    \end{definition}

    \begin{definition}
        \textbf{Metoda potęgowa} służy do wyznaczania \textbf{największej wartości własnej} macierzy. Korzysta z założenia:
        $\mathbf{x^{N+1} = A^{N+1} x^0}$.

        \begin{enumerate}
            \item W \textit{i}-tej iteracji: $x_i = Ay_{i-1}$, $y_i = \frac{x_i}{||x_i||_2}$
            \item \textit{i}-te przybliżenie największej wartości własnej: $\lambda_i = \frac{<x_i, y_{i-1}>}{<y_{i-1}, y_{i-1}>}$
        \end{enumerate}
    \end{definition}

    \begin{definition}
        \textbf{Odwrotna metoda potęgowa} służy do wyznaczenia \textbf{najmniejszej wartości własnej} macierzy.
        \begin{enumerate}
            \item $Ax_1 = x_0, Ax_i = x_{i-1}$
            \item \textit{i}-te przybliżenie najmniejszej wartości własnej: $\frac{1}{\lambda_i} = \frac{<x_i, x_{i-1}>}{<x_{i-1}, x_{i-1}>}$
        \end{enumerate}
    \end{definition}

    \begin{definition}
        \textbf{Metoda QR}  pozwala na przybliżone wyznaczenie \textbf{wszystkich wartości własnych} macierzy. Dla:
        \[A_i = Q_i R_i, ~~~~ A_{i+1} = R_i Q_i\]

        $A_i$ dąży do macierzy przekątniowej z wartościami własnymi na przekątnej.

        Q, R to macierze z \textbf{rozkładu QR}:
        \begin{gather*}
            A = QR, ~~ Q - \text{ortonormalna ($QQ^\mathcal{T}=I$)}, ~~ R - \text{górnotrójkątna}\\
            A = [a_1 \dots a_n], Q = [q_1 \dots q_n], q_1 = \frac{a_1}{||a_1||_2}, q_i = \frac{\tilde{q_i}}{||q_i||_2}\\
            \tilde{q_i} = a_i - \sum_{j=1}^{i-1} r_{ji} q_j, ~~ r_{ji} = \frac{<a_i, q_j>}{<q_j, q_j>}, r_{jj} = ||\tilde{q_j}||_2
        \end{gather*}
    \end{definition}

    \newpage


    \section{Interpolacja wielomianowa: metody Lagrange'a i Hermite'a. Efekt Rungego.}

    \subsection{Interpolacja wielomianowa}

    Interpolacja wielomianowa jest numeryczną \textbf{metodą przybliżania funkcji}.
    Wielomian $p$ o możliwie najniższym stopniu interpoluje wartości $y_k$ w węzłach $x_k$ gdy przyjmuje porządane wartości
    w odpowiednich punktach. Jeśli są to wartości pewnej funkcji $f$ to mówimy też że $p$ interpoluje $f$.

    \subsection{Wzór interpolacyjny Lagrange'a}

    Wyrażamy wielomian $p$ jako sumę
    \begin{align*}
        \mathbf{p(x)=\sum _{k=0}^{n}y_{i}l_k(x), ~~~~
        p(x)=\sum _{i=0}^{n}y_{i}\prod _{j=0\land j\neq i}^{n}{\frac {x-x_{j}}{x_{i}-x_{j}}}}.
    \end{align*}
    w której $l_k$ są wielomianami zależnymi od węzłów $x_0,x_1, \dots x_n$ ale nie od wartości $y_0,y_1, \dots ,y_n$.
    Gdyby jedna z nich, mianowicie $y_i$, była równa 1, a pozostałe by znikały, to mielibyśmy równość
    \begin{align*}
        \delta_{ij} = p_n(x_j)=\sum _{k=0}^{n}y_{i}l_k(x_j)=\sum _{k=0}^{n}\delta_k l_k (x_j)=l_i(x_j) \quad (0 \leq j \leq n)
    \end{align*}

    ($\delta_{ij}$ jest deltą Kroneckera). Łatwo znaleźć wielomian $l_i$ o tej własności. Jego zerami są wszystkie węzły
    oprócz $x_i$, czyli dla pewniej stałej c jest
    \begin{align*}
        l_i(x)=c(x-x_0)\dots(x-x_{i-1})(x-x_{i+1})\dots(x-x_n)
    \end{align*}
    a $c$ wynika z warunku $l_i(x_i)=1$.


    \subsection{Interpolacja Hermite’a}

    Interpolacja umożliwiająca znalezienie \textbf{wielomianu przybliżającego według wartości funkcji} na $n$ zadanych węzłach
    oraz na wartościach jej pochodnych.
    Węzeł zadany bez pochodnej jest węzłem pojedynczym, a węzeł z podanymi pochodnymi $k$-tego stopnia jest węzłem
    $k+1$-krotnym.

    \subsubsection{Algorytm}

    Budujemy tabelę z użyciem \textbf{różnic dzielonych} z tym wyjątkiem, że przy węzłach $k$-krotnych, $k>1$, gdzie,
    de facto, nie można obliczyć różnicy dzielonej, podstawia się $\frac{f^{(k)}(x_i)}{k!}$. (W tabeli przedstawiony jest
    $3$-krotny węzeł $x_{1}$).\\

    \textbf{Różnica dzielona} jest zdefiniowana rekurencyjnie:
    \begin{gather*}
        f[x_{i}]=f(x_{i}), ~~~
        f[x_{i},\ldots ,x_{i+j+1}]= \frac {f[x_{i+1},\ldots ,x_{i+j+1}]-f[x_{i},\ldots ,x_{i+j}]}{x_{i+j+1}-x_{i}}\\
    \end{gather*}


    \begin{tabular}{|c|c|cccll}
        \hline
        $x_i$ &
        $f(x_i)$ &
        \multicolumn{1}{c|}{$f[x_{i-1}]$} &
        \multicolumn{1}{c|}{$f[x_{i-2},x_{i-1},x_i]$} &
        \multicolumn{1}{c|}{$f[x_{i-3},x_{i-2},x_{i-1},x_i]$} &
        \multicolumn{1}{c|}{$\dots$} &
        \multicolumn{1}{c|}{$f[x_{i-n},\dots,x_i]$} \\ \hline
        $x_0$ &
        $f(x_0)$ &
        &
        &
        \multicolumn{1}{l}{} &
        &
        \\ \cline{1-3}
        $x_1$ &
        $f(x_1)$ &
        \multicolumn{1}{c|}{$f[x_0,x_1]$} &
        &
        \multicolumn{1}{l}{} &
        &
        \\ \cline{1-4}
        $x_1$ &
        $f(x_1)$ &
        \multicolumn{1}{c|}{$f'(x_1)$} &
        \multicolumn{1}{c|}{$f[x_0,x_1,x_1]$} &
        \multicolumn{1}{l}{} &
        &
        \\ \cline{1-5}
        $x_1$ &
        $f(x_1)$ &
        \multicolumn{1}{c|}{$f'(x_1)$} &
        \multicolumn{1}{c|}{$\frac{f''(x_1)}{2!}$} &
        \multicolumn{1}{c|}{$f[x_0,x_1,x_1,x_1]$} &
        &
        \\ \cline{1-6}
        \vdots &
        \vdots &
        \multicolumn{1}{c|}{$\vdots$} &
        \multicolumn{1}{c|}{$\vdots$} &
        \multicolumn{1}{c|}{$\vdots$} &
        \multicolumn{1}{c|}{$\ddots$} &
        \\ \hline
        $x_n$ &
        $f(x_n)$ &
        \multicolumn{1}{c|}{$f[x_{n-1},x_n]$} &
        \multicolumn{1}{c|}{$f[x_{n-2},x_{n-1},x_n]$} &
        \multicolumn{1}{c|}{$f[x_{n-3},x_{n-2},x_{n-1},x_n]$} &
        \multicolumn{1}{c|}{$\dots$} &
        \multicolumn{1}{c|}{$f[x_0,\dots,x_n]$} \\ \hline
    \end{tabular}\\\\

    Definiując $a_{i}$ jako wartości na przekątnej, $i=1,2 \dots ,m$ oraz $\bar {x}_{i} = \{x_j ~ : ~ j < i\}$.

    \begin{align*}
        w(x)=\sum _{i=0}^{m}a_{i}\prod _{j=0}^{i-1}(x-{\bar {x}}_{j}),
    \end{align*}


    \subsection{Efekt Rungego}

    \textbf{Efekt Rungego} - \textbf{pogorszenie jakości} interpolacji wielomianowej, mimo \textbf{zwiększenia liczby
    węzłów}.
    \begin{itemize}
        \item Pogorszenie szczególnie widoczne na końcach przedziałów.
        \item Typowe dla interpolacji za pomocą wielomianów wysokich stopni przy stałych odległościach węzłów oraz przy
        interpolowacji funkcji nieciągłych albo odbiegających znacząco od funkcji gładkiej.
        \item Dla uniknięcia efektu, stosuje się interpolację z węzłami coraz gęściej upakowanymi na krańcach przedziału
        interpolacji. Np. węzłami interpolacji n-punktowej wielomianowej powinny być miejsca zerowe wielomianu Czebyszewa
        n-tego stopnia.

    \end{itemize}

    \newpage


    \section{Zmienne losowe dyskretne. Definicje i najważniejsze rozkłady.}

    \begin{definition}
        \textbf{Zmienne dyskretne}.
        \begin{align*}
            P(x) = P(X=x)
        \end{align*}

        \textbf{Obliczanie prawdopodobieństwa}: $P(X \in A) = \sum_{x \in A} P(x)$.


        \textbf{Skumulowana funkcja rozkładu}: $F(x) = P(X \leq x) = \sum_{y \leq x}P(y)$.

        \textbf{Całkowite prawdopodobieństwo}: $\sum_{x}P(x) = 1$.

        \textbf{Wartość oczekiwana} (średnia): $EX = \sum_{x} x P(x)$.

        \textbf{Wariancja} (kwadrat odchylenia stand.): $VarX = \sigma^2 = E[ (X - \mu)^2 ]$.
    \end{definition}

    \begin{table}[H]
        \begin{center}
            \begin{tabular}{ p{3.0cm} | p{6cm}  p{0.8cm}  p{1cm}  p{3cm}}
                Rozkład & P(x) & EX & VarX\\
                \toprule

                \textbf{Bernoulli(p)} &
                \[P(x) = \left\{\begin{array}{lr}
                                    p, & \text{for } x = 1\\
                                    q = (1-p), & \text{for } x = 0
                \end{array}\right.\]
                &
                $p$
                &
                $pq$
                &
                próba\\
                \cmidrule(rl){2-5}

                \textbf{Binomial(n, p)} &
                $P(x) = \binom{n}{x} p^x (1-p)^{n-x} \text{ for } x = 0,1,\dots$
                &
                $np$
                &
                $npq$
                &
                liczba sukcesów z n prób\\
                \cmidrule(rl){2-5}

                \textbf{Geometric(p)} &
                $P(x) = (1-p)^{x-1}p \text{ for } x = 1,2,\dots$

                $P(X>k) = (1-p)^k$
                &
                $\frac{1}{p}$
                &
                $\frac{1-p}{p^2}$
                &
                liczba prób do sukcesu\\
                \cmidrule(rl){2-5}

                \textbf{Poiss($\lambda$)} &
                $P(x) = e^{-\lambda} \frac{\lambda^x}{x!} \text{ for } x = 0,1,\dots$
                &
                $\lambda$
                &
                $\lambda$
                &
                rozkład zdarzeń rzadkich\\
                \bottomrule
            \end{tabular}
        \end{center}
    \end{table}

    \begin{gather*}
        X_i \sim Bernoulli(p), S_n = \sum_{i=1}^{n} X_i, 0.05 \leq p \leq -0.95
    \end{gather*}

    \newpage


    \section{Zmienne losowe ciągłe. Definicje i najważniejsze rozkłady.}

    \begin{definition}
        \textbf{Zmienne ciągłe}.
        \begin{align*}
            f(x) = F'(x)
        \end{align*}

        \textbf{Obliczanie prawdopodobieństwa}: $P(X \in A) = \int_{A} f(x)dx$.

        \textbf{Skumulowana funkcja rozkładu}: $F(x) = P(X \leq x) = \int_{-\infty}^{x}f(y)dy$.

        \textbf{Całkowite prawdopodobieństwo}: $\int_{- \infty}^{\infty}f(x)dx = 1$.

        \textbf{Wartość oczekiwana}: $EX = \int xf(x) dx$.

        \textbf{Wariancja}: $VarX = \int_{- \infty}^{\infty} (x - \mu)^2 f(x) dx$.
    \end{definition}

    \begin{table}[H]
        \begin{center}
            \begin{tabular}{ p{2.7cm} | p{5.5cm} p{0.8cm} p{1cm} p{3.5cm}}
                Rozkład & f(x), F(x) & EX & VarX\\
                \toprule

                \textbf{Unif(a,b)} &
                $f(x) = \frac{1}{b-a} \text{ for } a \leq x \leq b$

                \[F(x) = \left\{\begin{array}{lr}
                                    0, & \text{for } x < a\\
                                    \frac{x-a}{b-a}, &  \text{for } a \leq x < b\\
                                    1, & \text{for } x \geq b
                \end{array}\right.\]
                &
                $\frac{a+b}{2}$
                &
                $\frac{(b-a)^2}{12}$
                &
                \\
                \cmidrule(rl){2-5}

                \textbf{Exp($\lambda$)} &
                $f(x) = \lambda e^{-\lambda x} \text{ for } x \geq 0$

                $F(x) = 1 - e^{-\lambda x}$
                &
                $\frac{1}{\lambda}$
                &
                $\frac{1}{\lambda^2}$
                & modelowanie czasu, brak pamięci\\
                \cmidrule(rl){2-5}

                \textbf{Gamma($\alpha, \lambda$)} &
                $f(x) = \frac{\lambda^{\alpha}}{\Gamma (\alpha)} x^{\alpha-1} e^{-\lambda x}$

                $F(x) = \frac{\lambda ^{\alpha}}{\Gamma(\alpha)} \int_{0}^{x} t^{\alpha-1} e^{-\lambda t} dt$
                &
                $\frac{\alpha}{\lambda}$
                &
                $\frac{\alpha}{\lambda ^2}$
                &
                łączny czas $\alpha$ niezależnych zdarzeń $\sim Exp(\lambda)$\\
                \cmidrule(rl){2-5}

                \textbf{N($\mu, \sigma$)} &
                $f(x) = \frac{1}{\sigma \sqrt{2 \pi}} e^{\frac{-(x - \mu)^2}{2 \sigma^2}}$

                $F(x) = \Phi(x)$ dla N(0,1)
                &
                $\mu$
                &
                $\sigma^2$
                &\\
                \bottomrule
            \end{tabular}
        \end{center}
    \end{table}

    \begin{gather*}
        T \sim Gamma(\alpha, \lambda), X \sim Poiss(\lambda t), ~~ P(T \leq t) = P( X \geq \alpha) \\
        Binomial(n, p) = N(np, \sqrt{np(1-p)})
    \end{gather*}


    \section{Łancuchy Markowa. Rozkład stacjonarny.}

    \begin{equation*}
        P =
        \begin{bmatrix}
            p_{1 1} & p_{1 2} & \dots & p_{1 n}\\
            p_{2 1} & p_{2 2} & \dots & p_{2 n}\\
            \dots & \dots & \dots & \dots\\
            p_{n 1} & p_{n 2} & \dots & p_{n n}
        \end{bmatrix}
    \end{equation*}
    \hfill \\

    \textbf{Rozkład w czasie h}: $P_h = P_0 * P^h$

    \textbf{Rozkład stacjonarny}: $\pi P = \pi$, $\sum \pi_i = 1$


    \section{Testy statystyczne: test z, test t-Studenta, test chi-kwadrat.}

    \begin{itemize}
        \item \textbf{Z-testów} używamy do sprawdzenia \textbf{czy} dana \textbf{próba pasuje do} zadanej
        \textbf{populacji} lub do  \textbf{porównywania dwóch dużych prób} (n $>$ 30).
        \item \textbf{T-testów} używamy do \textbf{porównywania dwóch małych prób} (n $<$ 30) testowych ze sobą,
        mogą również służyć do porównywania próby do zadanej wartości (np. średniej).
        Próby mogą być zależne lub niezależne.
        \item \textbf{Chi-kwadrat} używamy do ustalania \textbf{goodness of fit} dla próbki względem populacji lub do
        \textbf{zbadania niezależności}.
    \end{itemize}

    \textbf{T-testy}

    \begin{table}[H]
        \begin{center}
            \begin{tabular}{ p{.2\textwidth} |p{.3\textwidth} |p{.25\textwidth} |p{.25\textwidth}}
                \toprule
                Hipoteza zerowa & Warunki & Statystyka & Stopnie swobody\\
                \toprule

                $\mu = \mu_0$ & Rozmiar próby $n$;

                nieznana $\sigma$ & $t = \frac{\bar{X} - \mu_0}{\frac{s}{\sqrt{n}}}$ & $n-1$\\

                \cmidrule(r){1-4}

                $\mu_X - \mu_Y = D$ & Rozmiary prób $n$, $m$
                nieznane, równe $\sigma_X = \sigma_Y$ & $t = \frac{\bar{X} - \bar{Y} - D}{s_p\sqrt{\frac{1}{n}+\frac{1}{m}}}$
                & $n+m-2$\\

                \cmidrule(r){1-4}

                $\mu_X - \mu_Y = D$ & Rozmiary prób $n$, $m$;

                nieznane, różne $\sigma_X \neq \sigma_Y$ & $t = \frac{\bar{X} - \bar{Y} - D}{\sqrt{\frac{s_X^2}{n} + \frac{s_Y^2}{m}}}$
                & aproksymacja Satterthwaite\\

                \bottomrule
            \end{tabular}
        \end{center}
    \end{table}

    \newpage

    \textbf{Z-testy}

    \begin{table}[H]
        \begin{center}
            \begin{tabular}{ p{2.5cm} |p{2.5cm} |p{1.5cm} |p{4cm} |p{4cm}}
                \toprule
                Hipoteza zerowa & Parametr,

                estymator & \multicolumn{2}{c|}{jeśli $H_0$ jest prawdziwa:} & Statystyka \\
                \toprule

                $H_0$ & $\theta, \hat{\theta}$ & $E(\hat{\theta})$ & $Var(\hat{\theta})$ & $Z = \frac{\hat{\theta} - \theta_0}{\sqrt{Var(\hat{\theta})}}$\\

                \cmidrule(r){1-5}

                $\mu = \mu_0$ & $\mu$, $\bar{X}$ & $\mu_0$ & $\frac{\sigma^2}{n}$ & $Z = \frac{\bar{X} - \mu_0}{\frac{\sigma}{\sqrt{n}}}$\\

                \cmidrule(r){1-5}

                $p = p_0$ & $p$, $\hat{p}$ & $p_0$ & $\frac{p_0(1-p_0)}{n}$ & $Z = \frac{\hat{p} - p_0}{\sqrt{\frac{\hat{p}(1-\hat{p})}{n}}}$\\

                \cmidrule(r){1-5}

                $\mu_X - \mu_Y = D$ &
                $\mu_X - \mu_Y$,

                $\bar{X} - \bar{Y}$
                & $D$ & $\frac{\sigma_X^2}{n} + \frac{\sigma_Y^2}{n}$ & $Z = \frac{\bar{X} + \bar{Y} - D}{\sqrt{\frac{\sigma_X^2}{n} + \frac{\sigma_Y^2}{m}}}$\\

                \cmidrule(r){1-5}

                $p_1 - p_2 = D$ & $p_1 - p_2$,

                $\hat{p_1} - \hat{p_2}$
                & $D$ & $\frac{p_1(1-p_1)}{n} + \frac{p_2(1-p_2)}{m}$ &
                $Z = \frac{\hat{p_1} - \hat{p_2} - D}{\sqrt{\frac{\hat{p_1}(1-\hat{p_1})}{n} + \frac{\hat{p_2}(1-\hat{p_2})}{m}}}$\\

                \cmidrule(r){1-5}

                $p_1 = p_2$ & $p_1 - p_2$,

                $\hat{p_1} - \hat{p_2}$ & $0$ & $p(1-p)\left(\frac{1}{n} + \frac{1}{m}\right)$

                gdzie $p = p_1 = p_2$ & $Z = \frac{\hat{p_1} - \hat{p_2}}{\sqrt{\hat{p}(1-\hat{p})\left(\frac{1}{n} + \frac{1}{m}\right)}}$

                gdzie $\hat{p} = \frac{n\hat{p_1} + m\hat{p_2}}{n + m}$\\
                \bottomrule
            \end{tabular}
        \end{center}
    \end{table}

    \textbf{Rozkład obserwacji o rozkładzie normalnym i wspólnej wariancji $\sigma^2$}
    \begin{equation*}
        \frac{(n-1)s^2}{\sigma^2} = \sum_{i=1}^{n} \left ( \frac{X_i - \bar{X}}{\sigma} \right )^2 \sim Chi-square(n-1) \sim Gamma \left ( \frac{n-1}{2}, \frac{1}{2} \right )
    \end{equation*}

    Zatem przedział ufności:
    \begin{equation*}
        \left [ \frac{(n-1)s^2}{\chi^2_{\frac{\alpha}{2}}}, \frac{(n-1)s^2}{\chi_{1 - \frac{\alpha}{2}}^2} \right ]
    \end{equation*}

    \textbf{Testy Chi kwadrat}
    \begin{table}[H]
        \begin{center}
            \begin{tabular}{ p{.1\textwidth} |p{.2\textwidth} |p{.2\textwidth} |p{.2\textwidth} | p{0.3\textwidth}}
                \toprule
                $H_0$ &  $H_A$ & Test statistic & Rejection region & P-value \\
                \toprule
                \multirow{3}{*}{$\sigma^2 = \sigma_0^2$} & $\sigma^2 > \sigma_0^2$
                & \multirow{3}{*}{$\frac{(n-1)s^2}{\sigma_0^2}$} & $\chi_{obs}^2 > \chi_{\alpha}^2$ & $P{\chi^2 \geq \chi_{obs}^2}$\\
                & $\sigma^2 < \sigma_0^2$ & & $\chi_{obs}^2 < \chi_{\alpha}^2$ & $P{\chi^2 \leq \chi_{obs}^2}$\\
                & $\sigma^2 \neq \sigma_0^2$ & & $\chi_{obs}^2 \geq \chi_{\frac{\alpha}{2}}^2$
                or $\chi_{obs}^2 \leq \chi_{\frac{\alpha}{2}}^2$
                & $2\min(P{\chi^2 \geq \chi_{obs}^2}, P{\chi^2 \leq \chi_{obs}^2})$\\
                \bottomrule
            \end{tabular}
        \end{center}
    \end{table}

    \textbf{Statystyka Chi-kwadrat}
    \begin{equation*}
        \chi^2 = \sum_{k=1}^{N} \frac{(Obs(k) - Exp(k))^2}{Exp(k)}, R = [\chi_{\alpha}^2, +\infty], P = P{\chi^2 \geq \chi_{obs}^2}
    \end{equation*}
    Rule of thumb: $Exp(k) \geq 5 \text{ for all }k = 1, \dots, N$.\\

    \textbf{Test Chi-kwadrat niezależności A i B}
    \begin{equation*}
        \chi_{obs}^2 = \sum_{i=1}^k \sum_{j=1}^m \frac{(Obs(i,j) - \hat{Exp}(i,j))^2}{\hat{Exp}(i,j)}, \hat{Exp}(i,j) = \frac{(n_{i.})(n_{.j})}{n}
    \end{equation*}


    \section{Wzór Bayesa i jego interpretacja.}


    \begin{gather*}
        P(E|F) = \frac{P(E \cap F)}{P(F)} = \frac{P(F|E) P(E)}{P(F)} \\
        P(E_1 \cap \dots \cap E_n) = \prod_{i=1}^{n} P(X_i | X_1, \dots, X_{i-1}) \\
        P(E) = \sum_{i=1}^{n} P(E|F_i)P(F_i) \text{ dla } \bigcap_{i=1}^{n} F_i = \Omega
    \end{gather*}


    %    TODO - interpretacja?


    \section{Istnienie elementów odwrotnych względem mnożenia w strukturze $(Z_m, +, *)$ w zależności od liczby
    naturalnej $m$. Rozszerzony algorytm Euklidesa.}

    \begin{theorem}
        \textbf{Tożsamość Bézouta}

        Dla niezerowych liczb całkowitych $a$ oraz $b$ o największym wspólnym dzielniku d, istnieją liczby całkowite $x$ oraz $y$, nazywane współczynnikami Bézouta, które spełniają równanie
        \[ax + by = d\]
    \end{theorem}

    \begin{definition}
        \textbf{Elementy odwrotne modulo}

        W strukturze $(Z_m, +, *)$ element odwrotny względem mnożenia dla $x$ to taki $x^{-1}$, że:
        \[ x \cdot x^{-1} = 1 \  mod \  m \]
        Element odwrotny istnieje, jeśli $NWD(x, m) = 1$\\


        Aby obliczyć element odwrotny do $x$ w $Z_{m}$ należy użyć rozszerzonego algorytmu Euklidesa.
    \end{definition}
    \subsection{Algorytm Euklidesa}
    Oblicza największy wspólny dzielnik $a, b$:

    \begin{table}[H]
        \begin{center}
            \begin{tabular}{| p{0.5\textwidth} | p{0.5\textwidth} |}
                \hline
                z modulo: & z odejmowaniem:\\
                \hline
                while b != 0:
                \begin{itemize}
                    \item $r = a \% b$
                    \item $a, b = b, r$
                \end{itemize}
                return a

                &
                while a != b:
                \begin{itemize}
                    \item if a > b: a = a - b
                    \item else: b = b - a
                \end{itemize}
                return a\\
                \hline
            \end{tabular}
        \end{center}

    \end{table}


    \subsection{Rozszerzony algorytm Euklidesa}
    Oprócz obliczenia największego wspólnego dzielnika, algorytm znajduje także współczynniki Bézouta. \\
    Mamy następujące ciągi:
    \begin{gather*}
        r_0 = a, r_1 = b \quad r_{i + 1} = r_{i - 1} - q_{i}r_{i}\\
        x_0 = 1, x_1 = 0 \quad x_{i + 1} = x_{i - 1} - q_{i}x_{i}\\
        y_0 = 0, y_1 = 1 \quad y_{i + 1} = y_{i - 1} - q_{i}y_{i}
    \end{gather*}
    gdzie $q_{i}$ oznacza wynik całkowitego dzielenia $r_{i - 1}$ przez $r_i$. Możemy zauważyć, że $r_i = x_i * a + y_i * b$.

    \newpage


    \section{Ortogonalność wektorów w przestrzeni $\mathbb{R}_n$; związki z liniową niezależnością. Metoda ortonormalizacji Grama-Schmidta}

    \begin{definition}
        \textbf{Ortogonalność.}

        Dla wektorów $x, y$ w przestrzeni $\mathbb{R}_n$ z zadanym iloczynem skalarnym $\cdot$
        \[x \bot y \Leftrightarrow x \cdot y = 0 \]

    \end{definition}

    \begin{definition}
        \textbf{Liniowa niezależność}

        Wektory $x_1, x_2 ,\dots, x_n$ w przestrzeni $\mathbb{R}^n$ nazywamy \textbf{liniowo niezależnymi} jeśli
        \[a_1 * x_1 + a_2 * x_2 + \ldots + a_n * x_n = 0 \Rightarrow a_1 = a_2 = \ldots = a_n = 0\]

        W przestrzeni $R_n$ zbiór liniowo niezależnych nie może zawierać więcej niż $n$ wektorów.\\
        Każdy układ ortogonalny wektorów w przestrzeni $\mathbb{R}^n$ jest liniowo niezależny.

    \end{definition}

    \begin{definition}

        \textbf{Baza przestrzeni $\mathbf{\mathbb{R}^n}$}

        Zbiór wektorów $B \subset \mathbb{R}^n$ nazywamy \textbf{bazą przestrzeni} $\mathbb{R}^n$ jeśli:

        \begin{enumerate}
            \item Jest liniowo niezależny
            \item Generuje przestrzeń $\mathbb{R}^n$ tzn. każdy wektor $\in \mathbb{R}^n$ można zapisać jako kombinację liniową wektorów z $B$
        \end{enumerate}

        Każdy ortogonalny układ $n$ wektorów w $\mathbb{R}^n$ jest bazą.
    \end{definition}

    \begin{definition}
        \textbf{Ortonormalność}

        Zbiór wektorów $A$ przestrzeni $\mathbb{R}^n$ z zadanym iloczynem skalarnym $\cdot$ jest ortonormalny, jeśli
        \[ \forall x,y \in A  \quad   x \cdot y =  \begin{cases}
                                                       0, & x \ne y \\
                                                       1, & x = y
        \end{cases}\]

        Zatem wektory należące do $A$ są \textbf{wersorami}.
    \end{definition}

    \newpage
    \subsection{Ortonormalizacja Grama-Schmidta}

    Ortonormalizacja to proces przekształcenia układu niezależnych liniowo wektorów w układ wektorów ortonormalnych.

    \noindent \textbf{Projekcję wektora} $\mathbf{v}$ na wektor $\mathbf{u}$ definiujemy jako:
    \[\mathrm{proj}_\mathbf{u} \mathbf{v} = \frac{\mathbf{v} \cdot \mathbf{u}}{\mathbf{u} \cdot \mathbf{u}} \mathbf{u}\]

    \begin{enumerate}
        \item Dla zbioru wektorów $A = \{\mathbf{v}_1, \dots,  \mathbf{v}_n\}$ oblicz
        $\{{\mathbf{u}_1, \ldots, \mathbf{u}_n}\}$ gdzie
        \[\mathbf{u}_i = \mathbf{v}_i - \sum_{j=1}^{i-1} \mathrm{proj}_\mathbf{u_j} \mathbf{v}_i, ~~~~~~ \text{tzn:} ~~~
        \mathbf{u}_1 = \mathbf{v}_1, ~~ \mathbf{u}_2 = \mathbf{v}_2 - \mathrm{proj}_\mathbf{u_1} \mathbf{v}_2 ~~ \text{itd.}
        \]

        \item Wektory $\{{\mathbf{u}_1, \ldots, \mathbf{u}_n}\}$ podziel przez ich normy, uzyskując
        $\{{\mathbf{e}_1, \ldots, \mathbf{e}_n}\}$ - układ wektorów ortonormalnych.
    \end{enumerate}


    \section{Liczby Stirlinga I i II rodzaju i ich interpretacja.}

    \begin{definition}
        \textbf{Liczby Stirlinga I rodzaju}.\\
        Przez $c(n,k)$ ($[^n_k]$) oznaczamy \textbf{liczbę permutacji zbioru n-elementowego}, które mają \textbf{rozkład}
        na dokładnie \textbf{k cykli rozłącznych}.

        Dla dowolnego $n \geq 1$ mamy:
        \begin{enumerate}
            \item $c(0,0) = 1$
            \item $c(n,0) = c(0,n) = 0$
            \item $c(n,k) = c(n-1, k-1) + (n-1) * c(n-1, k)$
        \end{enumerate}
        W szczególności: $c(n,n) = 1, c(n,1) = (n-1)!, \sum^n_{k=0} c(n,k) = n!$.
    \end{definition}

    \begin{definition}
        \textbf{Liczby Stirlinga II rodzaju}.\\
        Przez $S(n,k)$ ($\{^n_k\}$) oznaczamy \textbf{liczbę rozmieszczeń n rozróżnialnych kul na k nierozróżnialnych
        stosach} w taki sposób, aby żaden \textbf{stos nie} był \textbf{pusty}.

        Dla dowolnego $n \geq 1$ mamy:
        \begin{enumerate}
            \item $S(0,0) = 1$,
            \item $S(n,0) = S(0,n) = 0$,
            \item $S(n,k) = S(n-1, k-1) + k \times S(n-1, k)$.
        \end{enumerate}
        w szczególności  $S(n,n) = S(n,1) = 1$.\\
    \end{definition}


    \section{Twierdzenia Eulera i Fermata; funkcja Eulera.}
    \begin{definition}
        \textbf{Funkcja Eulera (Tocjent)}

        Funkcja przypisująca każdej \textbf{liczbie naturalnej liczbę mniejszych liczb względnie pierwszych z nią}.
        \[\sum_{m|n}^{} \varphi(m) = n\]
    \end{definition}

    \textbf{Własności}
    \begin{enumerate}
        \item $\varphi (n) \leq n - 1$  dla każdego $n>1$
        \item $\varphi (p) \leq p - 1$ dla każdego p będącego liczbą pierwszą
        \item $\varphi (mn) = \varphi (m)\varphi(n)$ jeśli NWD(m,n) = 1 (m i n względnie pierwsze)
        \item $\varphi(p^{k}) = p^{k-1} \cdot (p-1)$ jeśli p jest liczbą pierwszą
        \item $\varphi(n) = n
        \left (  1 - \frac{1}{p_1}\right)
        \left (  1 - \frac{1}{p_2}\right)
        \cdots
        \left (  1 - \frac{1}{p_n}\right)$
        gdzie $p_1, p_2, \dots, p_n$ są czynnikami pierwszymi liczby n
        \item jeżeli $n = \prod_{i=1}^{k} p_i^{k_i}$ jest rozkładem liczby n na czynniki pierwsze $\varphi (n) = \prod_{i=1}^{k} \varphi \left( p_i^{k_i} \right )$
    \end{enumerate}

    \begin{definition}
        \textbf{Małe Twierdzenie Fermata}
        jeżeli p - pierwsza, to:

        \begin{enumerate}
            \item $a^{p-a} \equiv 0 \pmod p ~~ \forall a \in \mathbb{Z}$,
            \item $a^{p-1} \equiv 1 \pmod p ~~ \forall a : a, p ~ \text{względnie pierwsze}$.
        \end{enumerate}
    \end{definition}

    \begin{definition}
        \textbf{Twierdzenie Eulera}

        dla $m \in \mathbb{Z_{+}}$, $a \in \mathbb{Z}$ względnie pierwszych:
        \[ a^{\varphi (m)} \equiv 1 \pmod m \]
    \end{definition}


    \section{Konfiguracje i t-konfiguracje kombinatoryczne.}

    \begin{definition}
        \textbf{Konfiuracją kombinatoryczną} $B$ o parametrach $\mathbf{(n,k,r)}$ nazywamy \textbf{rodzinę k-elementowych
        podzbiorów} w X, jeżeli \textbf{każdy element} $x \in X$ występuje w dokładnie \textbf{r podbiorach}.
        \begin{enumerate}
            \item $n = |X|$
            \item $|B_i|=k \quad \forall i=1, \dots, b$
            \item $n \cdot r = b \cdot k ~~ \Rightarrow k | n\cdot r$
        \end{enumerate}
        Konfiguracja istnieje wtw. $\frac{n\cdot r}{k}\leq {n\choose k}$.


    \end{definition}

    \begin{definition}
        \textbf{T-konfiguracja} używa $r_t$ oznaczające ilość wystąpień t-elementowych podzbiorów X.
        \[r_{t-1} = r_t \cdot \frac{n-t+1}{k-t+1}\]

        Jeśli $B$ jest $t$-konfiguracją kombinatoryczną, to
        $B$ jest $s$-konfiguracją kombinatoryczną dla $s = 1, 2,\dots$ $t-1$.
        \[r_s = r_t \cdot \frac{(n-s)(n-s-1)\dots(n-t+1)}{(k-s)(k-s-1)\dots(k-t+1)}\]

        \textbf{Warunek konieczny} (nie wystarczający) dla istnienia t-konfiguracji ($t \geq 2$):
        \[(k-s)(k-s-1)\dots(k-t+1)|r_t(n-s)(n-s-1)\dots(n-t+1) \\ dla\  0\leq s \leq (t-1)\]
    \end{definition}


    \section{Cykl Hamiltona, obwód Eulera, liczba chromatyczna - definicje i twierdzenia.}

    \subsection{Cykl Hamiltona}
    \begin{itemize}
        \item \textbf{Ścieżka Hamiltona} to ścieżka przechodząca przez wszystkie wierzchołki.

        \item \textbf{Cykl Hamiltona} to nim cykl przechodzący przez wszystkie wierzchołki. Taki \textbf{graf}
        nazwyamy \textbf{hamiltonowski}m.

        \item Jeśli dla dowolonych \textbf{różnych, niesąsiednich} wierzchołków $u, v \in V$ zachodzi:
        \begin{itemize}
            \item $d(u) + d(v) \geq n$, to G jest \textbf{hamiltonowski} ($|V_G| \geq 3$).
            \item $d(u) + d(v) \geq n-1$, to G ma \textbf{ścieżkę hamiltona} ($|V_G| \geq 2$).
        \end{itemize}
        \item Jeśli \textbf{minimalny stopień} wierzchołka wynosi co najmniej $\mathbf{\frac{n}{2}}$, to G jest
        \textbf{hamiltonowski} ($|V_G| \geq 3$).
    \end{itemize}

    \subsection{Obwód Eulera}
    \begin{itemize}
        \item \textbf{Droga Eulera} to droga w której każda krawędź użyta jest dokładnie raz.

        \item \textbf{Obwód Eulera} to droga Eulera w której pierwszy i ostatni wierzchołek są identyczne.
        Graf ten nazywamy wtedy \textbf{eulerowskim}.

        \item Jeśli $G$ spójny to:
        \begin{gather*}
            \text{ G jest grafem \textbf{eurelowskim}} \Leftrightarrow \text{stopień każdego wierzchołka w G jest \textbf{parzysty}}\\
            \text{  G ma \textbf{drogę} Eulera } \Leftrightarrow \text {G ma dokładnie \textbf{0 lub 2} wierzchołki
            stopnia \textbf{nieparzystego}}
        \end{gather*}
    \end{itemize}

    \subsection{Liczba chromatyczna}

    \begin{itemize}
        \item \textbf{Kolorowanie wierchołkowe} grafu $G = (V, E)$ przy użyciu (co najwyżej) $k$
        kolorów to funkcja $c: V \rightarrow \{1, \dots, k\}$ spełniająca warunek
        \[c(u) \neq c(v) ~~ \forall u, v \in V : uv \in E\]

        \item \textbf{Liczba chromatyczna} to najmniejsza liczba kolorów dla
        której istnieje kolorwanie wierzchołkowe $G$. Oznaczamy $\chi(G)$.
        \begin{enumerate}
            \item $\chi(G) \leq d_{\max}(G) + 1$.
            \item $\chi(G) = 1 \Leftrightarrow |E(G)| = 0$
            \item $\chi(T) = 2$ dla każdego drzewa T o przynajmniej dwóch wierzchołkach
            \item $\chi(C_{2k}) = 2, \chi(C_{2k+1}) = 3$ dla grafu cyklicznego
            \item $\chi(K_n) = n$ dla grafu pełnego
        \end{enumerate}

        \item G spójny $\Rightarrow \chi(G) \leq d_{\max}(G)$, o ile G nie jest grafem pełnym ani cyklem o
        nieparzystej liczbie wierzchołków.
    \end{itemize}


    \section{Algorytm Forda-Fulkersona wyznaczania maksymalnego przepływu.}

    \begin{itemize}
        \item \textbf{Siecią przepływową} nazywamy graf skierowany z wyróżnionym \textbf{źródłem} i \textbf{ujściem} oraz
        \textbf{funkcją pojemności krawędzi}.

        \item \textbf{Przepływem} nazywamy funkcję $f$ spełniającą następujące warunki:
        \begin{enumerate}
            \item $\forall \vec{uv} \in \vec{E}: ~~ f(\vec{uv}) \leq c(\vec{uv})$,
            \item $\forall v \in V \backslash \{s, t\}: ~~ \sum_{\vec{uv} \in \vec{E}} ~ f(\vec{uv}) = \sum_{\vec{vw} \in \vec{E}} ~ f(\vec{vw})$.
        \end{enumerate}

        \item \textbf{Wartość przepływu} -- $|f| =  \sum_{\vec{su}  \in \vec{E}} f(\vec{su}) = \sum_{\vec{wt} \in \vec{E}} f(\vec{wt})$

        \item \textbf{Sieć rezydualna} to sieć zawierająca również krawędzie przeciwne do krawędzi w $\vec{E}$ z funkcją
        pojemności określoną wzorem:
        \begin{gather*}
            c_f(\vec{uv}) = c(\vec{uv}) - f(\vec{uv}) ~~ \text{dla} ~ \vec{uv} \in  \vec{E} ~  \text{t. że} ~ c(\vec{uv}) > 0\\
            c_f(\vec{uv}) = f(\vec{uv}) ~~ \text{dla} ~ \vec{uv} \notin  \vec{E} ~  \text{t. że} ~ \vec{vu} \in \vec{E} \text{i} f(\vec{uv}) > 0
        \end{gather*}
        oraz $c_f(\vec{uv}) = 0$ w pozostałych przypadkach (ile jeszcze może płynąć w przód/ile płynie wstecz).

        \item \textbf{Ścieżka rozszerzająca} dla przepływu f to dowolna ścieżkę skierowana łączącą wierzchołki
        źródło z ujściem krawędziami z niepełną pojemnością.

        \item \textbf{Algorytm Forda-Fulkersona} znajduje \textbf{maksymalny przepływ} w sieci:
        \begin{enumerate}
            \item przyjmujemy początkowo dowolny (np. zerowy) przepływ $f$,
            \item budujemy sieć rezydualną $G_f$,
            \item dopóki w $G_f$ istnieje ścieżka rozszerzająca $P$ określamy jej przepływ $c_p$ i uaktualniamy wartości
            funkcji $c_f$ sieci rezydualnej.
        \end{enumerate}

        \item \textbf{Przekrój} sieci przepływowej to para rozłącznych zbiorów $(S, T)$ dzielących $V$ na podzbiór z
        źródłem i podzbiór z ujściem. \textbf{Pojemność przekroju}:
        \[c(S, T) = \sum_{(u,v) \in \vec{E} \cap (S \times T)} c(\vec{uv})\]

        \item \textbf{Maksymalna wartość przepływu jest jest równa minimalnej pojemności przekroju}.
    \end{itemize}


    \section{Rozwiązywanie równań rekurencyjnych przy użyciu funkcji tworzących (generujących) oraz przy użyciu
    równania charakterystycznego.}

    \begin{definition}
        \textbf{Liniowym równaniem rekurencyjnym rzędu r} (o stałych współczynnikach) nazywamy równanie postaci:
        \begin{align*}
            x_{n+r} = c_1 x_{n+r-1} +  \cdots + c_r x_n + f(n), ~~~ c_i  \in \mathbb{R}, c_r \neq 0, f:  \mathbb{N} \rightarrow \mathbb{R}
        \end{align*}
        Jeśli funkcja $\forall n ~ f(n) = 0$, to powyższe równanie nazywamy \textbf{jednorodnym}.

        \textbf{Rozwiązani szczególne} to dowolny ciąg spełniający zależność rekurencyjną.

        \textbf{Rozwiązanie ogólne} to wzór ogólny pozwalający wyznaczyć wszystkie rozwiązania szczególne.
    \end{definition}

    \begin{definition}
        \textbf{Funkcja tworząca} (generujaca) ciągu $(a_n)$ to szereg
        \begin{align*}
            \sum_{n=0}^{\infty} a_n x^n.
        \end{align*}
        Jest to tylko inny zapis ciągu (wyraz $a_n$ to współczynnik przy $x^n$).
    \end{definition}

    \textbf{Operacje na funkcjach tworzących}.
    \begin{enumerate}
        \item Funkcje tworzące można \textbf{dodawać, odejmować i mnożyć przez liczbę}.
        \item \textbf{Mnożeniu dwóch funkcji tworzących} odpowiada operacja "splotu", tzn. jeśli
        $A(x) =  \sum_{n=0}^{\infty} a_n x^n$ i $B(x) = \sum_{n=0}^{\infty} b_n x^n$, to
        $A(x) * B(x) = \sum_{n=0}^{\infty} c_n x^n$, gdzie $c_n = \sum_{k=0}^n a_k b_{n-k}$.
        \item Jeśli funkcja tworząca A(x) ma "element odwrotny", tzn. taką funkcję tworzącą B(x) dla której
        $A(x) * B(x) = 1$, to mówimy że A(x) jest \textbf{odwracalna} i piszemy $\frac{1}{A(x)} = B(x)$.
        \item Funkcje tworzące można \textbf{różniczkować} według wzoru:
        \begin{align*}
            (\sum_{n=0}^{\infty} a_n x^n)' = \sum_{n=1}^{\infty} n a_n x^{n-1} = \sum_{n=0}^{\infty} (n+1) a_{n+1} x^n
        \end{align*}
        \item Zachodzą wzory na \textbf{pochodne} znane z analizy, np. $(A(x) * B(x))' = A'(x)B(x) + A(x)B'(x)$.
        \item Operacja \textbf{"całkowania"} (odwrotna do różniczkowania):
        \begin{align*}
            \int (\sum_{n=0}^{\infty} a_n x^n) = \sum_{n=0}^{\infty} \frac{a_n}{n+1} x^{n+1} = \sum_{n=0}^{\infty} \frac{a_n - 1}{n} x^n
        \end{align*}
        Wyraz $a_n$ można "odzyskać" z funkcji tworzącej A(x) wykonując jej n-krotne różniczkowanie.
    \end{enumerate}

    \begin{theorem}
        Zachodzi:
        Dla dowolnej liczby naturalnej $m \geq 1$ zachodzi wzór:
        \begin{gather*}
            \frac{1}{(1-x)^m} ~ = ~ \sum_{n=0}^{\infty} \binom{n+m-1}{n} x^n ~~~~ \forall m \in \mathbb{N}\\
            (1+x)^{\alpha} ~ = ~ \sum_{n=0}^{\infty} \binom{\alpha}{n} x^n ~~~~~~~~ \forall \alpha \in \mathbb{R}, |x| < 1
        \end{gather*}
    \end{theorem}

    \begin{definition}
        \textbf{Równanie charakterystyczne} równania rekurencyjnego. Weźmy równanie rekurencyjne jednorodne z danymi A, B:
        \begin{align*}
            a_n = A a_{n-1} + B a_{n-2}
        \end{align*}
        Załóżmy, że ma ono rozwiązanie postaci $a_n = t^n$. Podstawiając otrzymujemy:
        \begin{gather*}
            t^n = A t^{n-1} + B t^{n-2}\\
            t^2 = A t^1 + B\\
            t^2 - A t^1 - B = 0
        \end{gather*}
        Jest to równanie charakterystyczne. $r_1, r_2$ -- pierwiastki równania, $C, D$ -- stałe:

        \begin{align*}
            a_n = \left\{\begin{matrix}
                             C r_1^n + D r_2^n, ~~ r_1 \neq r_2\\
                             (C + Dn) r_1^n, ~~ r_1 = r_2
            \end{matrix}\right.
        \end{align*}
    \end{definition}


    \section{Ciąg i granica ciągu liczbowego, granica funkcji.}

    \subsection{Ciągi.}

    \begin{definition}
        \textbf{Ciąg liczbowy} -- funkcja $\mathbb{N} \rightarrow \mathbb{R}$, oznaczamy przez $(a_n)$, zbiór wartości
        $\{a_n\}$, n-ty wyraz $a_n$.
    \end{definition}

    \begin{definition}
        \textbf{Granice}.\\
        Granica \textbf{właściwa} ciągu: $lim_{n  \rightarrow \infty} ~ a_n = a, ~~ a \in \mathbb{R} \Leftrightarrow$
        \begin{gather*}
            \forall \varepsilon > 0 ~~ \exists  n_0 \in \mathbb{N} ~~ \forall n \in \mathbb{N} ~~~ [(n > n_0) \Rightarrow (|a_n - a| < \varepsilon)]
        \end{gather*}

        Granica \textbf{niewłaściwa} ciągu: $lim_{n \rightarrow \infty} ~ a_n = + (-) \infty \Leftrightarrow$
        \begin{align*}
            \forall \varepsilon > (<) 0 ~~ \exists  n_0 \in \mathbb{N} ~~ \forall n \in \mathbb{N} ~~~ [(n > n_0) \Rightarrow (a_n > (<) \varepsilon)]
        \end{align*}
    \end{definition}


    \begin{theorem}
        \textbf{Twierdzenia}.
        \begin{enumerate}
            \item \textbf{O ograniczoności ciągu zbieżnego}. Jeśli ciąg jest zbieżny do granicy właściwej, to jest ograniczony.

            \item \textbf{O równoważności granic:}
            $~~ lim_{n \rightarrow \infty} ~ a_n = 0 ~~ \Leftrightarrow ~~ lim_{n \rightarrow \infty} ~ |a_n| = 0.$

            \item \textbf{O dwóch ciągach}. Weźmy ciągi $(a_n), (b_n): a_n \leq b_n \forall n \geq n_0$, wtedy:
            \begin{gather*}
                lim_{n \rightarrow \infty} a_n = \infty ~~~ \Rightarrow lim_{n \rightarrow \infty} ~ b_n = \infty\\
                lim_{n \rightarrow \infty} b_n = -\infty ~~~ \Rightarrow lim_{n \rightarrow \infty} ~ a_n = -\infty
            \end{gather*}

            \item \textbf{O trzech ciągach}. Weźmy ciągi $(a_n), (b_n), (c_n): a_n \leq b_n \leq c_n \forall n \geq n_0$:
            \[lim_{n  \rightarrow \infty} a_n = lim_{n \rightarrow \infty} c_n = b \Rightarrow a_n \leq b_n \leq c_n ~~~ \forall n \geq n_0\]

            \item \textbf{O ciągu monotonicznym i ograniczonym}. Jeżeli ciąg $(a_n)$ jest niemalejący dla $n \geq n_0$ oraz
            ograniczony z góry, to jest zbieżny do granicy właściwej $\sup\{a_n : n \geq n_0\}$. Nierosnący analogicznie.
        \end{enumerate}
    \end{theorem}

    \subsection{Funkcje.}

    \begin{definition}
        \textbf{Heinego granicy właściwej funkcji w punkcie}. $x_0 \in \mathbb{R}$, $f$ -- określona na sąsiedztwie
        $S(x_0)$:
        \begin{gather*}
            lim_{x \rightarrow x_0} ~ f(x) = g\\
            \Leftrightarrow\\
            \forall_{(x_n): ~ \{x_n\} \subset S(x_0)} ~~ [(lim_{n \rightarrow \infty} x_n = x_0) \Rightarrow (lim_{n \rightarrow \infty} f(x_n) = g)].
        \end{gather*}
    \end{definition}

    \begin{definition}
        \textbf{Cauchy'ego granicy właściwej funkcji w punkcie}. $x_0 \in \mathbb{R}$, $f$
        określona przynajmniej na sąsiedztwie $S(x_0)$:
        \begin{gather*}
            lim_{x \rightarrow x_0} ~ f(x)  = g\\
            \Leftrightarrow\\
            \forall \varepsilon > 0 ~~ \exists \delta > 0 ~~ \forall  x \in S(x_0)  ~~~ [(|x - x_0| <  \delta) \Rightarrow (|f(x) - g| < \varepsilon)].
        \end{gather*}

        Granica niewłaściwa:
        \begin{gather*}
            lim_{x \rightarrow x_0} ~ f(x)  = \infty\\
            \Leftrightarrow\\
            \forall \varepsilon > 0 ~~ \exists \delta > 0 ~~ \forall  x \in S(x_0)  ~~~ [(|x - x_0| <  \delta) \Rightarrow (f(x) > \varepsilon)].
        \end{gather*}
    \end{definition}

    \begin{theorem}
        \textbf{Warunek konieczny i wystarczajacy istnienia granicy}.
        \begin{align*}
            lim_{x \rightarrow x^{-}_0}  ~ f(x) ~ = ~ lim_{x \rightarrow x^{+}_0}  ~ f(x)
        \end{align*}
    \end{theorem}


    \begin{theorem}
        \textbf{O nieistnieniu granicy funkcji w punkcie}. Jeśli istnieją ciągi $(x'_n)$, $(x''_n)$ spełniające warunki:
        \begin{enumerate}
            \item $lim_{n \rightarrow \infty} x'_n = x_0$, przy czym $x'_n \neq x_0 ~~ \forall n \in \mathbb{N}$
            oraz $lim_{n \rightarrow \infty}  ~ f(x'_n) = g'$,
            \item $lim_{n \rightarrow \infty} x''_n = x_0$, przy czym $x''_n \neq x_0 ~~ \forall n \in \mathbb{N}$
            oraz $lim_{n \rightarrow \infty}  ~ f(x''_n) = g''$,
            \item $g' \neq g''$,
        \end{enumerate}
        to granica $lim_{x \rightarrow x_0} ~ f(x)$ nie istnieje (właściwa ani niewłaściwa).
    \end{theorem}

    \begin{theorem}
        \textbf{O niesitnieniu granicy funkcji w nieskończoności}. Jeżeli istnieją ciągi $(x'_n)$, $(x''_n)$
        spełniające warunki:
        \begin{enumerate}
            \item $lim_{n \rightarrow \infty} ~ x'_n = \infty $ oraz $ lim_{n \rightarrow \infty} ~ f(x'_n) = g'$,
            \item $lim_{n \rightarrow \infty} ~ x''_n = \infty $ oraz $ lim_{n \rightarrow \infty} ~ f(x''_n) = g''$,
            \item $g' \neq g''$,
        \end{enumerate}
        to nie istnieeje granica $lim_{x \rightarrow x_0} ~ f(x)$ (właściwa ani niewłaściwa).
    \end{theorem}

    \begin{theorem}
        \textbf{O dwóch funkcjach}. Weźmy $f, g: f(x) \leq g(x) ~~ \forall x \in S(x_0)$, wtedy:
        \begin{gather*}
            lim_{x \rightarrow x_0} ~ f(x) = \infty \Rightarrow lim_{x \infty x_0} ~ g(x) = \infty
        \end{gather*}
    \end{theorem}

    \textbf{Granice specjalne}
    \setlength{\jot}{10pt}
    \begin{align*}
        &\lim_{x \to +\infty} a^{x} x^{\alpha} \stackrel{[0 \cdot \infty]}{=}  \text{0 dla a} \in (0, 1), \alpha \geq 0 \\
        &\lim_{x \to 0} \frac{\sin{x}}{x} \stackrel{[\frac{0}{0}]}{=} \text{1 oraz}    \lim_{x \to 0} \frac{\arcsin{x}}{x} \stackrel{[\frac{0}{0}]}{=} 1\\
        &\lim_{x \to 0} (1 + x)^{\frac{1}{x}} \stackrel{1^{\infty}}{=} \text{e oraz}   \lim_{x \to \infty} (1 + x)^{\frac{1}{x}} \stackrel{\infty^{0}}{=} 1 \\
        &\lim_{x \to 0} \frac{a^x -1}{x} \stackrel{[\frac{0}{0}]}{=} \ln{a} \text{ dla a $>$ 0, w szczególności } \lim_{x \to 0} \frac{e^x -1}{x} \stackrel{[\frac{0}{0}]}{=} 1 \\
        &\lim_{x \to 0} \frac{\log_{a}(1 + x)}{x} \stackrel{[\frac{0}{0}]}{=} \frac{1}{\ln{a}} \text{, w szczególności } \lim_{x \to 0} \frac{\ln(1 + x)}{x} \stackrel{[\frac{0}{0}]}{=} 1 \\
        &\lim_{x \to \pm \infty} (1 + \frac{a}{x})^{x} \stackrel{[1^{\infty}]}{=} e^{a} \text{, dla a} \in \mathbb{R} \\
        &\lim_{x \to 0} \frac{(1 + x)^a - 1}{x} \stackrel{[\frac{0}{0}]}{=} \text{a, dla a} \in \mathbb{R}
    \end{align*}

    \begin{theorem}
        \textbf{Reguła de L'Hostpiala}. Jeżeli funkcja $f$ i  $g$ spełniają warunki:
        \begin{enumerate}
            \item $lim_{x \rightarrow x_0} f(x) = lim_{x \rightarrow x_0} g(x) = 0 ~~ (\infty)$, przy czym $g(x) \neq 0$ dla $x \in S(x_0)$
            \item istnieje granica $lim_{x \rightarrow x_0} \frac{f'(x)}{g'(x)}$ (właściwa lub niewłaściwa),
        \end{enumerate}
        to
        \begin{align*}
            lim_{x \rightarrow x_0} \frac{f(x)}{g(x)} = lim_{x \rightarrow x_0} \frac{f'(x)}{g'(x)}
        \end{align*}
    \end{theorem}


    \section{Ciągłość i pochodna funkcji. Definicja i podstawowe twierdzenia.}

    \subsection{Ciągłość.}

    \begin{definition}
        \textbf{Funkcja ciągła w punkcie}. $x_0 \in \mathbb{R}$, $f$ określona przynajmniej na otoczeniu $O(x_0)$.
        \begin{align*}
            f \text{ciągła w }x_0 ~ \Leftrightarrow ~ lim_{x \rightarrow x_0} ~ f(x) = f(x_0).
        \end{align*}

        \textbf{Funkcja jest ciągła na zbiorze}, jeżeli jest ciągła w każdym punkcie tego zbioru.
    \end{definition}

    \begin{theorem}
        \textbf{Warunek konieczny i wystarczający ciągłości funkcji}. Funckja jest ciągła w punkcie wtw,
        gdy jest w tym punkcie ciągła lewostronnie i prawostronnie.
    \end{theorem}

    \begin{definition}
        \textbf{Nieciągłość funkcji}. $x_0 \in \mathbb{R}$, $f$ określona przynajmniej na otoczeniu $O(x_0)$.
        Funkcja $f$ jest nieciągła w punkcie $x_0$ wtedy i tylko wtedy, gdy nie istnieje
        granica $lim_{x \rightarrow x_0} ~ f(x)$ albo gdy $lim_{x \rightarrow x_0} ~ f(x) \neq f(x_0)$.
        \hfill \\

        \textbf{Nieciągłość pierwszego rodzaju}:
        \begin{align*}
            lim_{x \rightarrow x^{-}_0} ~ f(x) \neq  f(x_0) ~~~ \text{lub} ~~~ lim_{x \rightarrow x^{+}_0} ~ f(x) \neq f(x_0).
        \end{align*}
        \begin{enumerate}
            \item typu "skok": $lim_{x \rightarrow x^{-}_0} ~ f(x) ~ \neq ~ lim_{x \rightarrow x^{+}_0} ~ f(x)$.
            \item typu "luka": $lim_{x \rightarrow x^{-}_0} ~ f(x) ~ = ~ lim_{x \rightarrow x^{+}_0} ~ f(x) ~ \neq ~ f(x_0)$.
        \end{enumerate}

        \textbf{Nieciągłość drugiego rodzaju}. Jeżeli co najmniej jedna z granic
        \begin{align*}
            lim_{x \rightarrow x^{-}_0} ~ f(x), ~~ lim_{x \rightarrow x^{+}_0} ~ f(x)
        \end{align*}
        nie istnieje lub jest niewłaściwa.
    \end{definition}

    \begin{theorem}
        \textbf{Twierdzenia}
        \begin{enumerate}
            \item \textbf{Weierstrassa o ograniczoności funkcji ciągłej}. Jeżeli funkcja jest ciągła na przedziale domkniętym
            i ograniczonym, to jest na nim ograniczona.

            \item \textbf{Weierstrassa o osiąganiu kresów}. Jeżeli funkcja $f$ jest ciągła na przedziale domkniętym $[a, b]$,to
            \begin{align*}
                \exists c \in [a,b] ~~ f(c) = ~ inf_{x \in [a,b]} ~ f(x) ~~ \text{oraz} ~~ \exists d \in [a,b] ~~ f(d) = ~ sup_{x \in [a,b]} ~ f(x)
            \end{align*}

            \item \textbf{Darboux o przyjmowaniu wartości pośrednich}. Jeżeli funkcja $f$ jest ciągła na przedziale $[a,b]$ oraz
            spełnia warunek $f(a) < f(b)$, to
            \begin{align*}
                \forall w \in (f(a), f(b)) ~ \exists c \in (a,b) ~~ f(c) = w.
            \end{align*}
        \end{enumerate}
    \end{theorem}


    \subsection{Pochodna.}

    \begin{definition}
        \textbf{Iloraz różnicowy}. $x_0 \in \mathbb{R}$, $f$ określona przynajmniej na otoczeniu $O(x_0, r)$,
        gdzie $r > 0$. Ilorazem różnicowym odpowiadającym przyrostowi $0 < |\Delta x| < r$ nazywamy:
        \begin{align*}
            \frac{\Delta f}{\Delta x} \stackrel{def}{=} \frac{f(x_0 + \Delta x) - f(x_0)}{\Delta x}
        \end{align*}
    \end{definition}

    \begin{definition}
        \textbf{Pochodna właściwa funkcji}. $x_0 \in \mathbb{R}$, $f$ określona przynajmniej na $O(x_0)$. Pochodna
        właściwa funkcji $f$ w punkcie $x_0$:
        \begin{gather*}
            f'(x_0) ~ \stackrel{def}{=} ~ lim_{x \rightarrow x_0} ~ \frac{f(x) - f(x_0)}{x - x_0}\\
            f'(x_0) ~ \stackrel{def}{=} ~ lim_{\Delta x \rightarrow 0} ~ \frac{f(x_0 + \Delta x) - f(x_0)}{\Delta x}
        \end{gather*}
        Pochodne jednostronne definiuje się anlogiczne na jednostronnym otoczeniu.
    \end{definition}

    \begin{theorem}
        \textbf{Warunek konieczny istnienia pochodnej właściwej funkcji}. Jeżeli funkcja ma pochodną właściwą w punkcie,
        to jest ciągła w tym punkcie. Implikacja odwrotna nie jest prawdziwa.
    \end{theorem}

    \begin{definition}
        \textbf{Pochodna funkcji na zbiorze}. Funkcja ma pochodną właściwą na zbiorze wtedy i tylko wtedy, gdy ma pochodną
        właściwą w każdym punkcie tego zbioru.
    \end{definition}

    \begin{theorem}
        \textbf{Zastosowanie różniczki do obliczeń przybliżonych}. Jeżeli funkcja $f$ ma pochodną właściwą w punkcie
        $x_0$, to
        \begin{align*}
            f(x_0 + \Delta x) \approx f(x_0) + f'(x_0)\Delta x
        \end{align*}
    \end{theorem}

    \begin{theorem}
        \textbf{Rolle'a}. Jeśli funkcja $f$ spełnia warunki:
        \begin{enumerate}
            \item jest ciągła na $[a,b]$
            \item ma pochodną właściwą lub niewłaściwą na $(a,b)$,
            \item $f(a) = f(b)$,
        \end{enumerate}
        to istnieje punkt $c \in (a,b)$ taki, że:
        \begin{align*}
            f'(c) = 0.
        \end{align*}
    \end{theorem}

    \begin{theorem}
        \textbf{Lagrange'a}. Jeżeli funkcja $f$ spełnia warunki:
        \begin{enumerate}
            \item jest ciągła na $[a,b]$,
            \item ma pochodną właściwą lub niewłaściwą na $(a,b)$,
        \end{enumerate}
        to istnieje punkt $c \in (a,b)$ taki, że
        \begin{align*}
            f'(c) = \frac{f(b)-f(a)}{b-a}
        \end{align*}
    \end{theorem}


    \section{Ekstrema funkcji jednej zmiennej. Definicje i twierdzenia.}

    \begin{definition}
        \textbf{Minimum lokalne funkcji}. Funkcja $f$ ma w punkcie $x_0 \in \mathbb{R}$ minimum lokalne, jeżeli:
        \begin{align*}
            \exists \delta > 0 ~ \forall x \in S(x_0, \delta) ~~ f(x) \geq f(x_0).
        \end{align*}

        \textbf{Minimum lokalne jest właściwe}, jeżeli:
        \begin{align*}
            \exists \delta > 0 ~ \forall x \in S(x_0, \delta) ~~ f(x) > f(x_0).
        \end{align*}
        Analogicznie definiujemy \textbf{maksimum} lokalne (właściwe).
    \end{definition}

    \begin{theorem}
        \textbf{Fermata, warunek konieczny istnienia ekstremum}. Jeżeli funkcja $f$ ma esktremum lokalne w punkcie $x_0$,
        oraz istnieje pochodna $f'(x_0)$, to $f'(x_0) = 0$.
    \end{theorem}

    \begin{theorem}
        \textbf{I warunek wystarczający istnienia ekstremum}. Jeżeli funkcja $f$ spełnia warunki:
        \begin{enumerate}
            \item $f'(x_0) = 0$,
            \item $\exists \delta > 0
            \left\{\begin{matrix}
                       f'(x) > 0 ~~ \forall x \in  S(x^{-}_0, \delta), \\
                       f'(x) < 0 ~~ \forall x \in  S(x^{+}_0, \delta),
            \end{matrix}\right.$
        \end{enumerate}
        to w punkcie $x_0$ ma maksimum lokalne właściwe. Analogicznie dla minimum.
    \end{theorem}

    \begin{theorem}
        \textbf{II warunek wystarczający istnienia ekstremum}. Jeżeli funkcja $f$ spełnia warunki:
        \begin{enumerate}
            \item $f'(x_0) = f''(x_0) = \dots = f^{(n-1)}(x_0) = 0$,
            \item $f^{(n)}(x_0) < 0$,
            \item $n$ jest liczbą parzystą, gdzie $n \geq 2$,
        \end{enumerate}
        to w punkcie $x_0$ ma maksimum lokalne właściwe. Analogicznie dla minimum,
    \end{theorem}


    \section{Całka Riemanna funkcji jednej zmiennej.}

    \begin{definition}
        \textbf{Całka oznaczona Riemanna}. Niech funkcja $f$ będzie ograniczona na przedziale $[a,b]$. Całkę oznaczoną
        Riemanna z funkcji  $f$  na przedziale $[a,b]$ definiujemy wzorem
        \begin{align*}
            \int_{a}^{b} f(x) \,dx ~~ \stackrel{def}{=} ~~ lim_{\delta(P)  \rightarrow 0} \sum_{k=1}^{n} f(x^{*}_k) \Delta x_k,
        \end{align*}
        o ile po prawej stronie znaku równości granica jest właściwa oraz nie zależy od sposoby podziałów $P$ przedziału
        $[a,b]$ ani od sposobów wyboru punktów pośrednich $x^{*}_k$, gdzie $1 \leq k \leq n$. Ponadto przyjmujemy
        \begin{align*}
            \int_a^a f(x)\,dx ~ \stackrel{def}{=} ~ 0 ~~~~ \text{oraz} ~~~~ \int_b^a f(x)\,dx ~ \stackrel{def}{=} ~ - ~ \int_a^b f(x) \,dx ~~ \text{dla} ~ a < b
        \end{align*}
        Funkcję, dla której istnieje całka Riemanna, nazywamy całkowalną.
    \end{definition}

    \begin{theorem}
        \textbf{Warunek wystarczający całkowalności funkcji}. Jeżeli funkcja $f$ jest ograniczona na przedziale $[a,b]$
        i ma na tym przedziale skończoną liczbę punktów nieciągłości I rodzaju, to jest na nim całkowalna.
    \end{theorem}

    \begin{theorem}
        \textbf{Obliczanie całek przy pomocy sumy całkowej podziału równomiernego}. Jeżeli funkcja $f$ jest całkowalna
        na przedziale $[a,b]$, to
        \begin{align*}
            \int_{a}^b f(x) \, dx ~ = ~ lim_{n \rightarrow \infty} [\frac{b - a}{n} \sum_{k=1}^n f (a + k \frac{b - a}{n})]
        \end{align*}
    \end{theorem}

    \begin{theorem}
        \textbf{Newtona - Leibniza, główne twierdzenie rachunku całkowego}. Jeżeli funkcja $f$ jest ciągła na przedziale
        $[a,b]$, to
        \begin{align*}
            \int_a^b f(x) \,dx ~ = ~ F(b) - F(a) ~ = ~ [F(x)]_a^b,
        \end{align*}
        gdzie F oznacza dowolną funkcję pierwotną funkcji $f$  na tym przedziale.
    \end{theorem}


    \section{Pochodne cząstkowe funkcji wielu zmiennych; różniczkowalność i różniczka funkcji.}

    \begin{definition}
        \textbf{Pochodne cząstkowe}. Niech funkcja $f$ będzie określona przynajmniej na otoczeniu punktu $(x_0, y_0)$.
        Pochodną cząstkową pierwszego rzędu funkcji $f$ względem $x$ w punkcie $(x_0, y_0)$ określamy wzorem:
        \begin{align*}
            \frac{\partial f}{\partial x}(x_0, y_0)  \stackrel{def}{=} lim_{\Delta x \rightarrow 0} \frac{f(x_0 + \Delta x, y_0) - f(x_0, y_0)}{\Delta x}
        \end{align*}
    \end{definition}

    \begin{definition}
        \textbf{Funkcja różniczkowalna w punkcie}. Niech istnieją pochodne cząstkowe $\frac{\partial f}{\partial x}(x_0, y_0)$
        $\frac{\partial f}{\partial y}(x_0, y_0)$. Funkcja $f$ jest różniczkowalna w punkcie $(x_0, y_0)$  wtw, gdy:
        \begin{align*}
            \underset{(\Delta x,  \Delta y) \rightarrow (0,0)}{lim} \frac{f(x_0 + \Delta x, y_0 +  \Delta y) - f(x_0, y_0) - \frac{\partial f}{\partial x}(x_0, y_0)\Delta x - \frac{\partial f}{\partial y}(x_0, y_0) \Delta y}{\sqrt{\Delta x^2 +  \Delta y^2}} = 0
        \end{align*}
    \end{definition}

    \begin{definition}
        \textbf{Różniczka funkcji}. Niech funkcja $f$ ma pochodne cząstkowe pierwszego rzędu w punkcie $(x_0, y_0)$. Różniczką
        funkcji $f$ w punkcie $(x_0, y_0)$ nazywamy funkcję $df(x_0, y_0)$ zmiennych $\Delta x, \Delta y$ określoną
        wzorem:
        \begin{align*}
            df(x_0, y_0)(\Delta x, \Delta y) \stackrel{def}{=} \frac{\partial f}{\partial x}(x_0, y_0)\Delta x + \frac{\partial f}{\partial y}(x_0, y_0)\Delta y
        \end{align*}
    \end{definition}

    \begin{theorem}
        \textbf{Zastosowanie różniczki funkcji do obliczeń przybliżonych}. Niech funkcja $f$ ma ciągłe pochodne cząstkowe
        pierwszego rzędu w punkcie $(x_0, y_0)$. Wtedy
        \begin{align*}
            f(x_0 + \Delta x, y_0 + \Delta y) \approx f(x_0, y_0) + df(x_0, y_0)(\Delta x, \Delta y)
        \end{align*}
    \end{theorem}

    \newpage


    \section{Ekstrema funkcji wielu zmiennych. Definicje i twierdzenia.}

    \begin{definition}
        \textbf{Minimum lokalne funkcji dwóch zmiennych}.
        Funkcja $f$ ma w punkcie $(x_0, y_0)$ minimum lokalne, jeżeli istnieje jego otoczenie
        $O: \forall x, y \in O$:
        \begin{align*}
            f(x,y) \geq f(x_0, y_0)
        \end{align*}
        Przy ostrej nierówności mówimy o minimum lokalnym \textbf{właściwym}.

        Analogicznie definijuemy \textbf{maksimum} lokalne (właściwe).
    \end{definition}

    \begin{theorem}
        \textbf{Warunek konieczny istnienia ekstremum}. Jeżeli funkcja $f$ ma ekstremum lokalne w punkcie $(x_0, y_0)$
        i istnieją pochodne cząstkowe w tym punkcie, to:
        \begin{align*}
            \frac{\partial f}{\partial x}(x_0, y_0) = 0, ~~ \frac{\partial f}{\partial y}(x_0, y_0) = 0
        \end{align*}

        Funkcja może mieć ekstrema tylko w punktach, w których wszystkie jej pochodne cząstkowe pierwszego rzędu się
        zerują albo w punktach, w których choć jedna z nich nie istnieje.
    \end{theorem}

    \begin{theorem}
        \textbf{Warunek wystarczający istnienia ekstremum}. Niech funkcja $f$ ma ciągłe pochodne cząstkowe rzędu drugiego
        na otoczeniu punktu $(x_0, y_0)$ oraz niech
        \begin{enumerate}
            \item $\frac{\partial f}{\partial x}(x_0, y_0) = 0, \frac{\partial f}{\partial y}(x_0, y_0) = 0$
            \item $\det \begin{bmatrix}
                            \frac{\partial^2 f}{\partial^2 x}(x_0, y_0) & \frac{\partial^2 f}{\partial x \partial y}(x_0, y_0) \\
                            \frac{\partial^2 f}{\partial x \partial y}(x_0, y_0) & \frac{\partial^2 f}{\partial^2 y}(x_0, y_0)
            \end{bmatrix} > 0$
        \end{enumerate}
        Wtedy w punkcie $(x_0, y_0)$ funkcja $f$ ma ekstremum lokalne i jest to:
        \begin{enumerate}
            \item \textbf{minimum}, gdy $\frac{\partial^2 f}{\partial^2 x}(x_0, y_0) > 0$,
            \item \textbf{maksimum}, gdy $\frac{\partial^2 f}{\partial^2 x}(x_0, y_0) < 0$.
        \end{enumerate}
    \end{theorem}


    \section{Twierdzenie o zmianie zmiennych w rachunku całkowym; współrzędne walcowe i sferyczne.}

    \begin{definition}
        \textbf{Twierdzenie o zmianie zmiennych w rachunku całkowym}. Niech
        \begin{enumerate}
            \item odwzorowanie $ T: \begin{cases}
                                        x = \phi(u,v,w) \\
                                        y = \psi(u,v,w) \\
                                        z = \chi(u,v,w)
            \end{cases}$ przekształca różnowartościowo wnętrze obszaru regularnego $\Delta$ na wnętrze obszaru
            regularnego $V$,
            \item funkcje $\phi$, $\psi$, $\chi$ mają ciągłe pochodne cząstkowe rzędu pierwszego na pewnym zbiorze
            otwartym zawierającym obszar $\Delta$,
            \item funkcja $f$ jest ciągła na obszarze $V$,
            \item jakobian $J_T$ jest różny od zera wewnątrz obszaru $\Omega$.
        \end{enumerate}
        Wtedy
        \begin{align*}
            \iiint_V f(x,y,z)\,dx\,dy\,dz = \iiint_{\Omega} f(\phi(u,v,w), \psi(u,v,w), \chi(u,v,w))
            |J_T| \,du\,dv\,dw
        \end{align*}
        gdzie
        \begin{align*}
            J_T (u,v, w) \stackrel{def}{=} det \begin{bmatrix}
                                                \frac{\partial \phi}{\partial u}(u,v,w)  & \frac{\partial \phi}{\partial v}(u,v,w) & \frac{\partial \phi}{\partial w}(u,v,w)\\
                                                \frac{\partial \psi}{\partial u}(u,v,w)  & \frac{\partial \psi}{\partial v}(u,v,w) & \frac{\partial \psi}{\partial w}(u,v,w)\\
                                                \frac{\partial \chi}{\partial u}(u,v,w)  & \frac{\partial \chi}{\partial v}(u,v,w) & \frac{\partial \chi}{\partial w}(u,v,w)
            \end{bmatrix}
        \end{align*}
    \end{definition}

    Obszar regularny: złożony z obszarów normalnych, czyli "ogrodzonych" funkcjami ciągłymi.

    \begin{definition}
        \textbf{Współrzędne walcowe}.
        \begin{gather*}
            W: \begin{cases}
                   x = \varrho cos \varphi \\
                   y = \varrho sin \varphi \\
                   z = h
            \end{cases} ~~~~
            |J_W| = \varrho
        \end{gather*}

        $\varphi$ - miara kąta między rzutem promienia wodzącego punktu $P$ na płaszczyznę $xOy$, a dodatnią częścią
        osi $Ox$, $0 \leq \varphi \leq 2 \pi$ albo $-\pi < \varphi \leq \pi$\\

        $\varrho$ - odległość rzutu punktu na $xOy$ od punktu (0,0,0), $0 \leq \varrho < \infty$\\

        $h$ - oznacza odległość punktu $P$ od płaszczyzny $xOy$, $-\infty < h < \infty$
    \end{definition}

    \begin{definition}
        \textbf{Współrzędne sferyczne}.
        \begin{align*}
            S: \begin{cases}
                   x = \varrho cos \varphi cos \psi \\
                   y = \varrho sin \varphi cos \psi \\
                   z = \varrho sin \psi
            \end{cases} ~~~~
            |J_S| = \varrho^2 cos \psi
        \end{align*}

        $\varphi$ - oznacza miarę kąta między rzutem promienia wodzącego punktu $P$ na płaszczyznę $xOy$, a dodatnią częścią
        osi $Ox$, $0 \leq \varphi \leq 2 \pi$ albo $-\pi < \varphi \leq \pi$\\

        $\psi$ - oznacza miarę kąta między promieniem wodzącym punktu $P$, a płaszczyzną $xOy$, $-\frac{\pi}{2} \leq \psi \leq \frac{\pi}{2}$\\

        $\varrho$ - oznacza odległość punktu $P$ od początku układu współrzędnych, $0 \leq \varrho < \infty$
    \end{definition}
\end{document}